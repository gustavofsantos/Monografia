\documentclass[tcc,capa]{texufpel}

\usepackage[utf8]{inputenc} % acentuacao
\usepackage{graphicx} % para inserir figuras
\usepackage[T1]{fontenc}

\hypersetup{
    hidelinks, % Remove coloração e caixas
    unicode=true,   %Permite acentuação no bookmark
    linktoc=all %Habilita link no nome e página do sumário
}

\unidade{Centro de Desenvolvimento Tecnológico}
\curso{Engenharia de Computação}
\nomecurso{Bacharelado em Engenharia de Computação}
\titulocurso{Bacharel em Engenharia de Computação}

\title{Proposta de exploração da plataforma Ethereum como base a aplicações em Fog Computing}

\author{Santos}{Gustavo Fernandes dos}
\advisor[Prof.~Dra.]{Reiser}{Renata Hax Sander}
\coadvisor[Prof.~Dr.]{Pilla}{Maurício}
%\collaborator[Prof.~Dr.]{Aguiar}{Marilton Sanchotene de}

%Palavras-chave em PT_BR
\keyword{palavrachave-um}
\keyword{palavrachave-dois}
\keyword{palavrachave-tres}
\keyword{palavrachave-quatro}

%Palavras-chave em EN_US
\keywordeng{keyword-one}
\keywordeng{keyword-two}
\keywordeng{keyword-three}
\keywordeng{keyword-four}

\begin{document}

\renewcommand{\advisorname}{Orientadora}           %descomente caso tenhas orientadora
%\renewcommand{\coadvisorname}{Coorientadora}      %descomente caso tenhas coorientadora

\newcommand{\bchain}{\textit{blockchain} }
\newcommand{\Bchain}{\textit{Blockchain} }

\maketitle 

\sloppy

\fichacatalografica

\folhadeaprovacao

%Opcional
\begin{dedicatoria}
  Dedico\ldots bla blabla blablabla bla. Bla blabla blablabla bla.\\
  Bla blabla blablabla bla. Bla blabla blablabla bla.
\end{dedicatoria}

%Opcional
\begin{agradecimentos}
  Bla blabla blablabla bla.  Bla blabla blablabla bla.  Bla blabla blablabla
  bla.  Bla blabla blablabla bla.  Bla blabla blablabla bla.  Bla blabla
  blablabla bla.  Bla blabla blablabla bla.  Bla blabla blablabla bla.  Bla
  blabla blablabla bla.  Bla blabla blablabla bla.  Bla blabla blablabla bla.
  Bla blabla blablabla bla.  Bla blabla blablabla bla.  Bla blabla blablabla
  bla.  Bla blabla blablabla bla.  Bla blabla blablabla bla.  Bla blabla
  blablabla bla.  Bla blabla blablabla bla.  Bla blabla blablabla bla.  Bla
  blabla blablabla bla.  Bla blabla blablabla bla.
\end{agradecimentos}

%Opcional
\begin{epigrafe}
  Bla blabla blablabla bla.\\
  Bla blabla blablabla bla.\\
  Bla blabla blablabla bla.\\
  Bla blabla blablabla bla.\\
  Bla blabla blablabla bla.\\
  {\sc --- Fulano de Tal}
\end{epigrafe}

%Resumo em Portugues (no maximo 500 palavras)
\begin{abstract}
  resumo
\end{abstract}

\begin{englishabstract}%
  {Titulo do Trabalho em Ingles}
  
	abstract
\end{englishabstract}

%Lista de Figuras
\listoffigures

%Lista de Tabelas
\listoftables

%lista de abreviaturas e siglas
\begin{listofabbrv}{SPMD}
        \item[IOT] Internet of Things
        \item[IPFS] Interplanetary File System
\end{listofabbrv}

%Sumario
\tableofcontents

\chapter{Introdução}

\section{Organização da Monografia}

% Como estruturar essa parte?
% Falar sobre moedas virtuais?

%=====================================================%

\chapter{Revisão Bibliográfica}

\section{Trabalhos Relacionados}


\chapter{Revisão Técnica}

\section{Blockchain}

\Bchain vêm atraindo a atenção de pessoas em um ritmo crescente, em especial após o ápice do Bitcoin no final do ano de 2017...

	\subsection{Arquitetura}
    
    A arquitetura da \bchain tem inicio pelo bloco primordial, chamado de \textit{genesis}. Os demais blocos que compõem uma \bchain consistem basicamente de duas partes: dados de transações e valor de hash. Dados de transações consistem em informações que os participantes inserem na \bchain e o valor de hash é uma referência ao bloco diretamente anterior ao bloco atual, é um dado gerado durante o processo de mineração.

	%\subsection{Histórico}

	\subsection{Obtenção do consenso}
    
    	\subsubsection{Prova de trabalho}
        
        \subsubsection{Prova de participação}
        
        \subsubsection{Prova de armazenamento}
        
        \subsubsection{Prova de existência}
        
        \subsubsection{...}
        
        % falar de outras provas de consenso?

	\subsection{Estudo de caso: Bitcoin}
    
    	\subsubsection{Arquitetura do Bitcoin}
        
        \subsubsection{Mineração}
        
        \subsubsection{Valor do Bitcoin}

\section{Ethereum}

	\subsection{Histórico}

	\subsection{Conceitos}
    
    	\subsubsection{Contas}
    
    	\subsubsection{Gas}
        
        \subsubsection{Smart contracts}

	\subsection{Ethereum Virtual Machine}
    
    \subsection{Mineração}
    
    	\subsubsection{Prova de participação}
        
        \subsubsection{Otimização}
    
    \subsection{Limitações}
    
    	\subsubsection{O preço da computação}
        
        \subsubsection{Ethereum como um banco de dados}

	\subsection{Segurança}
    
    \subsection{Aplicações}
    
\section{Interplanetary File System}

	\subsection{Sistemas de arquivos distribuídos}
    
    \subsection{Sistemas de arquivos descentralizados}

	\subsection{Arquitetura}
    
    	\subsubsection{Identidade}
        
        \subsubsection{Rede}
        
        \subsubsection{Rotas}
        
        \subsubsection{Trocas}
        
        \subsubsection{Objetos}
        
        \subsubsection{Arquivos}
        
        \subsubsection{Sistema de nomes}
    
    \subsection{Limitações}
    
    	\subsubsection{Censura de conteúdo}
        
    \subsection{Aplicações}

\section{Internet das Coisas}

	\subsection{Segurança}

\section{Fog Computing}

	\subsection{Cloud Computing}
    
    \subsection{Fog Computing e a Internet das Coisas}


%=====================================================%

\chapter{Modelagem}

%=====================================================%
%              Fim da monografia parcial              %
%=====================================================%

\chapter{Desenvolvimento}

%=====================================================%

\chapter{Resultados}

%=====================================================%

\chapter{Conclusão}



% Bibliografia http://liinwww.ira.uka.de/bibliography/index.html um
% site que cataloga no formato bibtex a bibliografia em computacao
% \bibliography{nomedoarquivo.bib} (sem extensao)
% \bibliographystyle{formato.bst} (sem extensao)

\bibliography{bibliografia} 
\bibliographystyle{abnt}
%\bibliographystyle{plain}

% Anexos (Opcional)
\annex
\chapter{Um Anexo}

texto

\chapter{Outro Anexo}

texto

\end{document}

